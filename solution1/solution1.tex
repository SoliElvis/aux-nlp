
\documentclass{article}

\usepackage{comment}
\usepackage[english]{isodate}
\usepackage{graphicx}
\usepackage[margin=1in]{geometry}
\usepackage{siunitx}
\usepackage{paracol}
\usepackage{enumitem}
\usepackage{bbm}
\usepackage[utf8]{inputenc}
\usepackage[T1]{fontenc}
\usepackage[bookmarks=true]{hyperref}
\usepackage{bookmark}
\usepackage{pdfpages} %\includepdf[pages={1}]{myfile.pdf}

\usepackage{amsmath}
\usepackage{amssymb}
\usepackage{amsthm}
\usepackage{mdsymbol}%for perp with two bars

\usepackage{mathtools,xparse}
\newtheorem{theorem}{Theorem}

\DeclarePairedDelimiter{\abs}{\lvert}{\rvert}
\DeclarePairedDelimiter{\norm}{\lVert}{\rVert}

\newcommand{\E}{\mathbb{E}}
\newcommand{\Var}{\mathrm{Var}}
\newcommand{\Cov}{\mathrm{Cov}}
\newcommand\given[1][]{\:#1\vert\:}


\sisetup{output-decimal-marker = {,}}
\newcommand*{\ft}[1]{_\mathrm{#1}} 
\newcommand*{\dd}{\mathop{}\!\mathrm{d}}
\newcommand*{\tran}{^{\mkern-1.5mu\mathsf{T}}}%transpose of matrix
\newcommand{\trace}{\mathrm{trace}}

%%new
%\newcommand{\tab}{\hspace{.2\textwidth}}
%\newcommand{\span}{\mathrm{Span}}
\renewcommand{\baselinestretch}{1.5}



%%%indenting
%\setlength{\tindent}{\parindent}
%\setlength{\parindent}{0pt}
%\renewcommand{\indent}{\hspace*{\tindent}}


\renewcommand{\familydefault}{\sfdefault}
\usepackage{titling}
\usepackage[backend=biber,citestyle=authoryear,style=ieee]{biblatex}
\usepackage{amsmath}
\usepackage{fancyhdr}
\pagestyle{fancy}

\title{Solutionnaire 1}
\author{Frederic Boileau}
\date{\today}



\begin{document}
\fancyhf{}
\fancyhead[L]{Dynamic programing}
\maketitle
\thispagestyle{fancy}

J'introduis quelques notations. En analyse convexe on dénote l'ensemble
des fonctions convexes \emph{propres} sur un espace euclidien $\mathbb E$
(dans notre cas simplement $\mathbb R^n$) par $\Gamma (\mathbb E)$.
\begin{equation*}
    \Gamma \triangleq \{f : \mathbb R^n \rightarrow \mathbb R \cup \{+\infty\}\}
\end{equation*}
Une fonction \emph{propre} est une fonction pour laquelle il existe au moins un
$x$ t.q. $f(x) < + \infty $ et pour laquelle pour tout $x$ $f(x) > -\infty$. 
On écarte ainsi certains cas pathologiques.

J'utilise la notation de Evans pour la d\'erivation. Ainsi $D_x f$ 
est la d\'eriv\'ee de f par rapport \`a x, si c'est un nombre, un vecteur
(comme pour le gradient) ou une matrice (jacobien) d\'epend du contexte
(de $f$ et de $x$) dans tous les cas $D$ est toujours une approximation
linéaire de f. 

Soit A une matrice. Si A est d\'efinie positive  (resp.\  semi-d\'efinies
positives) on \'ecrit $A \succ 0$ (resp $A \succeq 0$).

\section{Direction de Descente}

Soit $f$ une fonction convexe différentiable sur $\mathbb R^n$ 
et $x$ et $y$ deux points de $\mathbb R^n$ t.q. $f(y) < f(x)$.
Montrer que $y-x$ est une direction de descente de $f$ en $x$.
Notons qu'aucun ordre n'est impos\'e  sur $x$ et $y$.\\

La preuve d\'ecoule directement de la d\'efinition de la 
convexit\'e donn\'ee en cours:
\begin{align*}
    f(\alpha y + (1-\alpha)x) &\leq \alpha f(y) + (1-\alpha)f(x)\\
    f(x + \alpha(y-x)) &\leq \alpha f(y) + (1-\alpha)f(x)\\
    f(x + \alpha(y-x)) &< \alpha f(x) + (1-\alpha)f(x)\\
    f(x + \alpha(y-x)) &< f(x)
\end{align*}

\clearpage

\section{Syst\`eme lin\'eaire et optimisation}
Prouvez que tout système linéaire $Ax = b$, avec A définie positive, peut se
reformuler comme un problème d’optimisation sans contrainte. 

Pour la reformulation nous avons plusieurs crit\`eres; premi\`erement on veut
que le minimum global corresponde \`a la solution, dans un second temps
si le probl\`eme est convexe nous avons des garanties quant \`a la recherche
de ce minimiseur global. Il y a deux formulations simples, je traite les
deux.
\subsection{Forme Quadratique}
Le reformulation la plus simple est de prendre la forme quadratique suivante:\\
\begin{equation}
    f(x) = \frac{1}{2}x\tran A x - x\tran b
\end{equation}

Le calcul de d\'eriv\'es par rapport \`a des vecteurs ou des matrices
peut \^etre laborieux mais l'identit\'e suivante est tr\`es importante:
\begin{equation}
    D_x (x\tran A x) = x\tran (A + A\tran)
\end{equation}
Ainsi nous avons:
\begin{equation}
    D_x(f) = Ax - b \qquad D_x^2(f) = A \succ 0
\end{equation}
La premi\`ere \'equation montre que le gradient de la fonction est nul ssi le
$x$ est une solution du syst\`eme lin\'eaire. La deuxi\`eme \'equation montre
que la deuxi\`eme d\'eriv\'ee (le hessien) de $f$ par rapport est
\emph{d\'efinie positive} (par supposition sur A). La fonction est donc
strictement convexe et atteint son minimum ssi le gradient est nulle, ce qui
n'arrive que lorsque $Ax = b$. \\

Notons que pour une fonction \`a une variable $D^2_x(f) > 0$ implique
que $f$ est strictement convexe en $x$. Pour une fonction \`a plusieures
variables, $f : \mathbb R^n \rightarrow \mathbb R$, nous avons que $D^2_x(f) =
\nabla^2_x (f) \succ 0$ implique $f$ strictement convexe. Nous voyons donc
que la notation est appropri\'ee puisque l'on g\'en\'eralise la notion de
`plus grand que z\'ero.'


\end{document}

